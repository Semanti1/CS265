
\chapter{Embedding Code in Text}
\label{embeddingcode} 

\section{Inlining Code in Text}
\label{inlinecode}

Use the  \texttt{ \textbackslash{}texttt\{\textit{code}\} } to inline a bit of code
in a paragraph.

You'll need to change text colors yourself.

\subsection{ Define Your Own Command }

You can define your own:

\begin{quote}
	\texttt{
		\textbackslash{}newcommand\{\textbackslash{}code\}[1]\{\textbackslash{}texttt\{\#1\}\}
	}
\end{quote}

\newcommand{\code}[1]{\texttt{#1}}

So you could then use the new command in-line 
\texttt{ \textbackslash{}code\{\textit{code}\} }

Honestly, not sure about this yet.  We'll get there.


\section{ Blocks of code }
\label{codeblock}

\subsection{ \texttt{verbatim} Environment }
\label{verbatim}

You can place a code block in a \texttt{verbatim} environment.  It seems that
all (most? many?) metacharacter behaviors are inhibited, and line breaks are
preserved.  I'm not so sure about leading white space.

\texttt{
	\textbackslash{}begin\{verbatim\}
		...
	\textbackslash{}end\{verbatim\}
}

Let's see how this looks:

\begin{verbatim}
   #include <stdio.h>

   char *name = "Kurt" ;

   int main( int argc, char *argv[] )
   {
      printf( "Hello, %s!\n", name ) ;

      return( 0 ) ;
   }  /* main */
\end{verbatim}

Hmmm.  Okay.  Moving on...


\subsection{ The \texttt{listings} Package }
\label{ listingspkg }

The \texttt{listings} package is part of \LaTeX{}'s standard library, I
believe.  It knows many languages (I hope), converts tabs to spaces you specify,
and, with the \texttt{color} package, will highlight literals, keywords,
comments, etc.  I do not yet know if you can define your own languages.

Here is an example, found on StackOverflow (see comment):

% This was ganked from
% http://stackoverflow.com/questions/3175105/how-to-insert-code-into-a-latex-doc

\begin{quote}
	\begin{verbatim}
   \usepackage{listings}
   \usepackage{color}

   \definecolor{dkgreen}{rgb}{0,0.6,0}
   \definecolor{gray}{rgb}{0.5,0.5,0.5}
   \definecolor{dkred}{rgb}{1, 0.6, 0.6}

   \lstset{frame=tb,
      language=C,
      aboveskip=3mm,
      belowskip=3mm,
      showstringspaces=false,
      columns=flexible,
      basicstyle={\small\ttfamily},
      numbers=none,
      numberstyle=\tiny\color{blue},
      keywordstyle=\color{dkgreen},
      commentstyle=\color{gray},
      stringstyle=\color{dkred},
      breaklines=true,
      breakatwhitespace=true,
      tabsize=3
   }
	\end{verbatim}
\end{quote}

You can, of course, define and use whichever colors you like.  Change default
language in the middle of document with
\texttt{\textbackslash{}lstset\{language=\textit{lang} \} } .

Then place your code in a \texttt{lstlisting} environment.  Let's see if we can
make our previous example better:


\begin{lstlisting}
	#include <stdio.h>

	char *name = "Kurt" ;

	int main( int argc, char *argv[] )
	{
		printf( "Hello, %s!\n", name ) ;

		return( 0 ) ;
	}  /* main */
\end{lstlisting}

You can even pull code right from a file, using

\begin{quote}
	\texttt{ \textbackslash{}lstinputlisting[language=Python]\{hello.py\} }
\end{quote}

%\lstset{language=Python}
%\lstinputlisting{hello.py}
\lstinputlisting[language=Python]{hello.py}

Nice.  Let's see if this is a better way to embed LaTeX.  There's a
\texttt{TeX} language listed.

\begin{lstlisting}[language=TeX]
You can even pull code right from a file, using \texttt{lstinputlisting}

	\lstinputlisting[language=Python]{hello.py}

So, I suppose this is for indented math expressions (note the super and
subscripts in math mode)

	\[ |f(y) - f(x)| < \epsilon \]
	\[ p_{i} = p_{i-1}^2 \]

\end{lstlisting}

Not exciting, but it works.


\subsection{Language Supported by \texttt{listings}}

This is according to \url{%
http://en.wikibooks.org/wiki/LaTeX/Source_Code_Listings#Supported_languages
} .  I don't see a date, know how current it is.

\begin{multicols}{4}
	\begin{itemize}
	\renewcommand{\labelitemi}{ }
		\item ABAP
		\item ACSL
		\item Ada
		\item Algol
		\item Ant
		\item Assembler
		\item Awk
		\item bash
		\item Basic
		\item C
		\item C++
		\item Caml
		\item Clean
		\item Cobol
		\item Comal
		\item csh
		\item Delphi
		\item Eiffel
		\item Elan
		\item erlang
		\item Euphoria
		\item Fortran
		\item GCL
		\item Gnuplot
		\item Haskell
		\item HTML
		\item IDL
		\item inform
		\item Java
		\item JVMIS
		\item ksh
		\item Lisp
		\item Logo
		\item make
		\item Mathematica
		\item Matlab
		\item Mercury
		\item MetaPost
		\item Miranda
		\item Mizar
		\item ML
		\item Modelica
		\item Modula-2
		\item MuPAD
		\item NASTRAN
		\item Oberon-2
		\item OCL
		\item Octave
		\item Oz
		\item Pascal
		\item Perl
		\item PHP
		\item Plasm
		\item PL/I
		\item POV
		\item Prolog
		\item Promela
		\item Python
		\item R
		\item Reduce
		\item Rexx
		\item RSL
		\item Ruby
		\item S
		\item SAS
		\item Scilab
		\item sh
		\item SHELXL
		\item Simula
		\item SQL
		\item tcl
		\item TeX
		\item VBScript
		\item Verilog
		\item VHDL
		\item VRML
		\item XML
		\item XSLT
	\end{itemize}
\end{multicols}

Take a peek at the link, above, there are notes I didn't bother with.
Various dialects of some of the languages are understood.

I was feelin' pretty good about the number of languages I'm familiar with,
'til I saw this list.  I've not even heard of some of these.

