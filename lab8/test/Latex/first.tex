
% use pdflatex to compile

\documentclass[a4paper,12pt,titlepage]{article}
\pagestyle{plain}
\title{First LaTeX Attempt}
\author{Kurt Schmidt\\
  Dept. of Computer Science,\\
  Drexel University\\
  3141 Chestnut St.\\
  Philadelphia, PA  19104}


%\\date{Dec 2012}usepackage{graphicx}
\usepackage[pdftex]{graphicx}
\graphicspath{{.}{~/public\_html/images}}

\begin{document}
\maketitle

% This is lifted largely from http://www.maths.tcd.ie/~dwilkins/LaTeXPrimer/
%

This is a paragraph.  My first one in \LaTeX.  They just need to be
separated by blank lines.

These are meta characters, so, need to be escaped:  \{, \}, \$, \_,
\%, \#, \& 

Backslash is special:  \textbackslash{}textbackslash

We have a hyphen -, endash --, and an emdash --- .

We have \emph{emphasis},

The \$ puts you into mathematics mode in a paragraph of regular text:  
$ f: R \to R $

So, I suppose this is for indented math expressions (note the super and
subscripts in math mode)

\[ |f(y) - f(x)| < \epsilon \]
\[ p_{i} = p_{i-1}^2 \]

In text mode, the \textasciicircum is for character composition: \^O

Left-- and right-- quotes are distinguished.  They are formed with 2
back-ticks, ``, and single-quotes, ''.  "Let's see what regular
double-quotes yield.  Ah.  Left quotes."

Some common control characters:

% this acts as a line comment
$\to$
$\rightarrow$
$\leftarrow$

Note, the lin% another comment
		 	   e comment is more; it also ignore the newline, and any leading
whitespace on the next line.

\{\} are used for grouping:  \emph{This is the emphasis tag.}

Let's see if we can get a superscript on a subscript:  $T_{n^2}$

\section{This is a Section}
\subsection{This is a Subection}
\subsection{And Another}

\section*{This Section Doesn't Have a Number}
\subsection*{Does this subsection?}
\subsection{Ah, unnumbered sections do not affect the numbering.}

\subsection*{Font Families (Typefaces)}
Well, \textrm{This is Roman family}
\textsf{This is Sans Serif family}
\texttt{This is typewriter (true type) family}

\subsection*{Font Shapes}
We have:  \textup{upright text (default)}
\textit{ Italics }
\textsl{Slanted, somehow different than italics?}, and
\textsc{We have cool small-caps, too.  Nice.}

\subsection*{Font Sizes}
\tiny{This is tiny}

\scriptsize{This is scriptsize}

\footnotesize{This is footnotesize}

\small{This is small}

\normalsize{This is normalsize}

\large{This is large}

\Large{This is Large}

\LARGE{This is LARGE}

\huge{This is huge}

\Huge{This is Huge}


\subsection*{Accents and Such through Control Sequences}
\normalsize{I dunno, those Size things seemed to persist}

S\'{\i}, math\'ematique

Hasta Ma\~nana, Pi\c{e}kos. Kurt G\"odel.  We have dotless {\i}s and {\j}s

There be others, of course.

\section*{Math}

Common control chars:  $\times$ $\div$ $\circ$ $\cap$ $\cup$ $\dagger$
$\ddagger$ $\otimes$ $\exists$ $\forall$ $\neg$ $\vert$ $\flat$ $\natural$
$\sharp$

\subsection*{Numbering Equations}

	\begin{equation}
		x + y = 17 
	\end{equation}
	\begin{equation}
		y = 3x - 42
	\end{equation}

\subsection*{Functions}

\[ \cos(\theta + \phi) = \cos \theta \cos \phi
      - \sin \theta \sin \phi \]

We can create ones not in the list: $\operatorname{foo } x$

Embedding Text in Math

\[ M^\bot = \{ f \in V' : f(m) = 0 \mbox{ for all } m \in M \}.\]

\subsection*{Fractions \& Roots}

The roots of a quadratic polynomial $a x^2 + bx + c$ with
$a \neq 0$ are given by the formula
\[ x = \frac{-b \pm \sqrt{b^2 - 4ac}}{2a} \]

The roots of a cubic polynomial of the form $x^3 - 3px - 2q$
are given by the formula
	\[ \sqrt[3]{q + \sqrt{ q^2 - p^3 }}
		+ \sqrt[3]{q - \sqrt{ q^2 - p^3 }} \]
where the values of the two cube roots must are chosen
so as to ensure that their product is equal to $p$.

We have 2 types of ellipsis\ldots

	\[ \frac{1 - x^{n+1}}{1 - x} = 1 + x + x^2 + \cdots + x^n \]

\subsection*{Brackets \& Norms}

Left and right (), \{\}, [], of appropriate size:
	\[ \left| 4 x^3 + \left( x + \frac{42}{1+x^4} \right) \right|.\]

	\[ \left. \frac{du}{dx} \right|_{x=0}.\] 

\subsection*{Multiline Formulae}

\begin{eqnarray}
\cos 2\theta & = & \cos^2 \theta - \sin^2 \theta \\
             & = & 2 \cos^2 \theta - 1.
\end{eqnarray}

The asterisk suppresses equation numbering:

If $h \leq \frac{1}{2} |\zeta - z|$ then
	\[ |\zeta - z - h| \geq \frac{1}{2} |\zeta - z|\] 
and hence
	\begin{eqnarray*}
	\left| \frac{1}{\zeta - z - h} - \frac{1}{\zeta - z} \right|
	& = & \left|
	\frac{(\zeta - z) - (\zeta - z - h)}{(\zeta - z - h)(\zeta - z)}
	\right| \\  & = &
	\left| \frac{h}{(\zeta - z - h)(\zeta - z)} \right| \\
		& \leq & \frac{2 |h|}{|\zeta - z|^2}.
	\end{eqnarray*}


\subsection*{Matrices}

The \emph{characteristic polynomial} $\chi(\lambda)$ of the
$3 \times 3$~matrix
\[ \left( \begin{array}{ccc}
a & b & c \\
d & e & f \\
g & h & i \end{array} \right)\] 
is given by the formula
\[ \chi(\lambda) = \left| \begin{array}{ccc}
\lambda - a & -b & -c \\
-d & \lambda - e & -f \\
-g & -h & \lambda - i \end{array} \right|.\]


\emph{Now} we can describe piecewise-defined functions (remember, mbox is
used to embed text in math):

	\[ |x| = \left\{
		\begin{array}{ll}
			 x & \mbox{if $x \geq 0$}\\
			-x & \mbox{if $x < 0$}
		\end{array} \right. \] 


\subsection*{Derivatives, Limits, Sums and Integrals}

Derivatives, now that we know fractions, are nothing special:

	\[ \frac{\partial u}{\partial t}
		 = h^2 \left( \frac{\partial^2 u}{\partial x^2}
      + \frac{\partial^2 u}{\partial y^2}
      + \frac{\partial^2 u}{\partial z^2} \right) \]

Limits and such:

	\[ \lim_{x \to +\infty}, \inf_{x > s} \mbox{and} \sup_K  \]

Sums:  $ \sum_{k=1}^m k^2 = \frac{m (m+1) (2m+1)}{6} $

Think we have products, too?  $\prod_{i=1}^m i = m! $.  Cool.

And we have integrals:
    \[ \int_0^{+\infty} x^n e^{-x} \,dx = n!.\] 

    \[ \int \cos \theta \,d\theta = \sin \theta + c.\] 

    \[ \int_{x^2 + y^2 \leq R^2} f(x,y)\,dx\,dy
       = \int_{\theta=0}^{2\pi} \int_{r=0}^R
          f(r\cos\theta,r\sin\theta) r\,dr\,d\theta.\] 

In some multiple integrals (i.e., integrals containing more than one integral
sign) one finds that LaTeX puts too much space between the integral signs. The
way to improve the appearance of of the integral is to use the control sequence
$\!$ to remove a thin strip of unwanted space

	\[ \int_0^1 \! \int_0^1 x^2 y^2\,dx\,dy. \] 

	\[ \int \!\!\! \int_D f(x,y)\,dx\,dy. \]


\section*{Lists}

We have ordered lists:

\begin{enumerate}
	\item \emph{Geno's}, not Pat's
	\item Order steak:
		\begin{enumerate}
			\item Say ``Steak''
			\item Say ``with'', if you want onions
			\item Tell which type of cheese
		\end{enumerate}
	\item You don't have all night, gotta eat it before it gets cold
\end{enumerate}

\noindent And bulleted lists:

\begin{itemize}
	\item Toothbrush
	\item Knife
	\item Bathing suit
	\item
		Documents
		\begin{itemize}
			\item passport
			\item dive log
		\end{itemize}
	\item Sunglasses
\end{itemize}

\noindent Finally, descriptive lists:

\begin{description}
	\item[halyard]
		Line that raises a sail
	\item[sheet]
		Line that trims a sail
	\item[reefing line]
		Line rigged to (effectively) shorten a sail
	\item[furler]
		Line that furls a sail (if it's so rigged).  E.g., if the main furls
		around the mast, you leave the halyard be, release the furling line, and
		use the outhaul to "put up" (pull out) the main, and the furler to pull
		it back in.  Convenient, but the sail is, necessarily, flatter than we'd
		like it to be.  Works much better for a foresail.
\end{description}

\section*{Quotes}

Use quote for shorter quotes:

\begin{quote}
Anti-intellectualism has been a constant thread winding its way through our
political and cultural life, nurtured by the false notion that democracy
means that `my ignorance is just as good as your knowledge.'

--Isaac Asimov
\end{quote}

Use quotation for longer quotes.  This is from the story ``Flight'', by John
Steinbeck.

\begin{quotation}
The dawn came and the heat of the day fell on the earth, and still Pepe slept.
Late in the afternoon his head jerked up. He looked slowly around. His eyes
were slits of weariness. Twenty feet away in the heavy brush a big tawny
mountain lion stood looking at him. Its long thick tall waved gracefully; its
ears were erect with interest, not laid back dangerously. The lion squatted
down on its stomach and watched him.

Pepe looked at the hole he had dug in the earth. A half-inch of muddy water had
collected in the bottom. He tore the sleeve from his hurt arm, with his teeth
ripped out a little square, soaked it in the water and put it in his mouth.
Over and over he filled the cloth and sucked it.

Still the lion sat and watched him. The evening came down but there was no
movement on the hills. No birds visited the dry bottom of the cut. Pepe looked
occasionally at the lion. The eyes of the yellow beast drooped as though he
were about to sleep. He yawned and his long thin red tongue curled out.
Suddenly his head jerked around and his nostrils quivered. His big tail lashed.
He stood up and slunk like a tawny shadow into the thick brush.

A moment later Pepe heard the sound, the faint far crash of horses' hoofs on
gravel. And he heard something else, a high whining yelp of a dog.
\end{quotation}

\section*{Tables}

The first five International Congresses of Mathematicians
were held in the following cities:
\begin{quote}
\begin{tabular}{lll}
Chicago&U.S.A.&1893\\
Z\"{u}rich&Switzerland&1897\\
Paris&France&1900\\
Heidelberg&Germany&1904\\
Rome&Italy&1908
\end{tabular}
\end{quote}

We can even get some borders:

\begin{tabular}{|r|r|}
\hline
$n$&$n!$\\
\hline
1&1\\
2&2\\
3&6\\
4&24\\
5&120\\
6&720\\
7&5040\\
8&40320\\
9&362880\\
10&3628800\\
\hline
\end{tabular}

\section{Embedding Images}

Note the graphicx package, included above.

Sadly, the type of image you embed depends upon your target type.  If the
final output is a Postscript (using dvips), then embed only PostScript
(Encapsulated PostScript) images.  If the target is PDF, then you can
include PDF, PNG, JPEG, or GIF.

\begin{figure}[htp]
	\centering
		\includegraphics[angle=30,scale=0.6]{sunFun.jpeg}
		%\includegraphics[scale=0.6]{sunFun.jpeg}
		\caption{A fun example picture}
		\label{fig:sun 1}
\end{figure}

%\begin{figure}[tph!]
%	\centerline{\includegraphics[angle=30,scale=0.6]{sunFun.jpeg}
%		%\includegraphics[scale=0.6]{sunFun.jpeg}
%		\caption{A fun example picture}
%		\label{fig:sun 2}
%\end{figure}

\end{document}

