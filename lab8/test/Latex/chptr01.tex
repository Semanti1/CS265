
\chapter{In a Galaxy, Far Far Away}
\label{starwars} % So I can \ref{altrings} later.

\section{Notes}
\label{notes}
%\input{chap1/sec11}

I'm just working my way through \LaTeX.  I put some Latex code here, and as
I gain proficiency, perhaps I'll put more.  For now, it would be helpful to
look at the source, as you read the book.  They should be in the same
directory.

Note, if the margins look odd on alternate pages, this is a \texttt{book}
document, meant to be bound.

For a text, this style of paragraph indentation/spacing is not so ideal.
I'll play later, but, for now, try playing with \texttt{
\textbackslash{}setlength\{\textbackslash{}parindent\}[\textit{width}] } and
\texttt{
\textbackslash{}setlength\{\textbackslash{}parskip\}[\textit{width}] }

Read up on \emph{rubber lengths}.  Also, be careful using \texttt{parskip}, as
it affects spacing in lists and other places.  The \texttt{parskip} package
might be helpful here.

\section{Heather Graham}
\label{graham}
%\input{chap1/sec12}
Talented, lovely.  And just gets more so.

\section{Compiling}

There should be a makefile in this directory, but I'm not at all happy with
it.

Also, I'm having issues using CYGWIN\_NT\-6.1 XXXXX 1.7.9(0.237/5/3)
2011-03-29 10:10 i686 Cygwin.  Packages are missing, compilation is somehow
more painful.  But, it's a pretty old install, so.  Ah, versions:

\begin{verbatim}
	$ latex --version
	pdfeTeX 3.141592-1.21a-2.2 (Web2C 7.5.4)
	kpathsea version 3.5.4
\end{verbatim}

Oh, did ya catch some approximation of $\pi$ in there?

On tux, we have:

\begin{verbatim}
	$ latex --version
	pdfTeX 3.1415926-1.40.10-2.2 (TeX Live 2009/Debian)
	kpathsea version 5.0.0
\end{verbatim}

Much newer version, anyway.  I should maybe update this thing.

Anyway, two ways I've been compiling TEX to PDF.  The first is two steps,
TEX$\rightarrow$DVI, then DVI$\rightarrow$PDF:

\begin{verbatim}
	$ latex book.tex
	...
	$ dvipdf book.dvi
	...
\end{verbatim}

The second way accomplishes the task in a single step:

\begin{quote}
	\texttt{
		\$ pdflatex book.tex
		...
	}
\end{quote}

Hmmmm.  \texttt{pdflatex} isn't creating a table of contents for me on tux,
either.  Oh!  Run it a couple times in succession.  \texttt{bibtex} might be
in there somewhere.

Also, \texttt{pdflatex} will allow you to use PDF-specific commands, and
seems to be recommended.

Note, if using references, indices, table of contents, etc., pay attention
to the output.  A 2\textsuperscript{nd} run might be suggested.

\subsection{Compiler Warnings and Errors}

You'll see various informational warnings:

\begin{quote}
	\texttt{
		LaTeX Font Warning: Font shape `OMS/cmtt/m/n' undefined
		...
	}
\end{quote}

You want to make sure that a correct substitution was made, but these are
fairly harmless.

Errors, of course, need to be corrected, and will leave you at an
interactive prompt (until I figure out how to signal batchmode).  I find
\bf{\texttt{x}} or \bf{\texttt{q}} to be helpful.

