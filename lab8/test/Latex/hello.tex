% hello.tex - simple example using the article
%
% 
\documentclass[a4paper,12pt,titlepage]{article}
\pagestyle{plain}
\title{Hello, World}
\author{Kurt Schmidt \\
	Drexel, Computer Science}
\date{Sept. 2014}

\begin{document}
\maketitle

Here's your obligatory `hello':  ``Hello.  Welcome to \LaTeX.''  \TeX is the
basic language, Developed by Donald Knuth.  \LaTeX is a way handy extension
to \TeX.  So, basic syntax probably applies to both.  Once we get into
packages, I've not clue, as yet, so, I'll just be talking about \LaTeX.

Easiest way to compile to PDF is using \texttt{pdflatex}.

\section{Math}
\label{hellomath}

And now, some math.  Remember, $\sum_{i=1}^m i = \frac{m(m+1)}{2}$, along
with identities for sums, and the sum of a geometric series.  You'll be
needing them.

Also recall these gems, you'll be needing them, too:

\[ b^{\log_{b}{x}} = x \]
\[ \log_{b}{b^x} = x \]
\[ \log_{b}{xy} = \log_{b}{x} + \log_{b}{y} \]
\[ \log_{b}{x/y} = \log_{b}{x} - \log_{b}{y} \]
\[ \log_{b}{x^n} = n\log_{b}{x} \]

So, 

\[ x^{\log_{b}{y}} = y^{\log_{b}{x}} \]

\section*{El Fin du Monde}
\label{end}

And that's it for now.  See section \ref{hellomath}.

\end{document}
