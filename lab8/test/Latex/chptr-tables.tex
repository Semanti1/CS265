
\chapter{Tables}
\label{tables} % So we can \ref{tables} later.

I took this mostly from http://www.andy-roberts.net/writing/latex/tables .

\section{Simple Tables}
\label{simple}

Introduce tables with the \emph{tabular} environment, describing columns.

\begin{quote}
	\texttt{ \textbackslash{}begin\{tabular\}\{ l c r \} }
\end{quote}

\begin{center}
	\begin{tabular}{ l r }
		l & left justified \\
		c & centered \\
		r & right justified \\
	\end{tabular}
\end{center}

\subsection{Vertical Lines}

Use the $\vert$ or $\Vert$ in the declaration for vertical separators.

\begin{center}
	\begin{tabular}{ l || r | }
		l & left justified \\
		c & centered \\
		r & right justified \\
	\end{tabular}
\end{center}

\subsection{Content of Tables}

Cells in a record are separated by \texttt{\&} .  
\textbackslash{}\textbackslash{} to start a new record.

Finally, use \texttt{\textbackslash{}hline} to put horizontal lines in:

\begin{center}
	\begin{tabular}{ l | r | }
		\hline 
		l & left justified \\
		\hline 
		c & centered \\
		\hline 
		r & right justified \\
		\hline 
	\end{tabular}
\end{center}


\section{Wrapping Text}
Sadly, \LaTeX doesn't wrap long lines in tables, by default.  It does
provide 3 other column specifiers which take a width argument, in
\texttt{pt}, \texttt{in}, \texttt{cm}, \texttt{mm}, or \texttt{em}:

\begin{center}
	\begin{tabular}{ | l | p{8cm} | }
		\hline
		Character & Meaning \\
		\hline
		\texttt{p\{$width$\}}
			& Paragraph column with text aligned vertically at the top \\
		\hline
		\texttt{m\{$width$\}}
			& Paragraph column with text vertically aligned in the middle (requires
			array package) \\
		\hline
		\texttt{b\{$width$\}}
			& Paragraph column with text vertically aligned at the bottom (requires
			array package) \\
		\hline
	\end{tabular}
\end{center}

The declaration for this table looks like:

\begin{quote}
	\texttt{ \textbackslash{}begin\{tabular\}\{ $\vert$ l $\vert$ p\{8cm\} $\vert$ \} }
\end{quote}

So, you'll need to play around a little, get it to come out nice.  Or, there
are packages already developed which'll do much of the work for you.

\section{Aligning on the Radix}
\label{radalign}

Again, there are packages that'll do this for you.  But, we're here, so...

\subsection{@-Expressions}
\label{atexpr}

Generally, the @ specifier takes a text argument, and a width specifier.
When appended to a column, it supresses normal cell spacing, and inserts the
text before each cell's contents.

We're going to use it without any space, to just use the radix to join two
columns of numbers (the integer and fractional parts).

\subsection{Tables of Numbers}

\begin{center}
	\begin{tabular}{ r @{.} l }
		123 & 456 \\
		3&14159265358979\\
		98765432 & 1 \\
	\end{tabular}
\end{center}


\section{Spanning}
\label{tablespan}

\subsection{Cell Spanning Multiple Columns}

In the data, place:

\begin{quote}
	\texttt{ \textbackslash{}multicolumn\{$numcols$\}\{$alignment$\}\{$contents$\} }
\end{quote}

, where $numcols$ is the number of subsequent columns (the width of this
cell, in columns), $alignment$ is one of \texttt{l}, \texttt{c}, or
\texttt{r} (maybe \texttt{p}?), with vertical separators, and, finally, the
contents of the cell.

\begin{center}
	\begin{tabular}{|r|l|}
		\hline
		\multicolumn{2}{|c|}{Endless Summer} \\
		\hline
		LOA & 43'5" \\
		\hline
		LWL & 36'4" \\
		\hline
		Beam & 12'10" \\
		\hline
		Draught & 5'11" \\
		\hline
		Displacement & 19,620 lbs \\
		\hline
		Fuel (diesel) & 63 gal \\
		\hline
		H$_2$O & 153 gal \\
		\hline
	\end{tabular}
\end{center}

\subsection{Cell Spanning Multiple Rows}

A cell spanning multiple rows is introduced with:

\begin{quote}
	\texttt{ \textbackslash{}multirow\{$numrows$\}\{$width$\}\{$contents$\} }
\end{quote}

To use this, you'll need the \texttt{multirow} package:

\begin{quote}
	\texttt{ \textbackslash{}usepackage\{multirow\} }
\end{quote}


, where $width$ could be a fixed width, or just use \* for the natural
width.

\begin{center}
	\begin{tabular}{|l|r|l|}
		\hline
		\multicolumn{3}{|c|}{Endless Summer} \\
		\hline
		\multirow{7}{*}{Specs}
			& LOA & 43'5" \\
			& LWL & 36'4" \\
			& Beam & 12'10" \\
			& Draught & 5'11" \\
			& Displacement & 19,620 lbs \\
			& Fuel (diesel) & 63 gal \\
			& H$_2$O & 153 gal \\
			\hline
		\multirow{5}{*}{Accomodations}
			& Cabins & 3 \\
			& Double Berths & 4 \\
			& Single Berths & 0 \\
			& Heads & 2 \\
			& Showers & 3 \\
			\hline
		\multirow{3}{*}{Rigging}
			& Masts & 1 \\
			\hline
			& \multicolumn{2}{|l|}{Furling main} \\
			& \multicolumn{2}{|l|}{Furling genoa} \\
			\hline
	\end{tabular}
\end{center}

Huh.  That \textbackslash{}hline went all the way through. I dunno.  Either
choose a package to help you make tables, or try embedding a table in the
table.

\section{Don't know yet}
\label{hold}

Whistling.
